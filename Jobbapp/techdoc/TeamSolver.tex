\documentclass[a4paper, 11pt]{article}

\usepackage{listings}
\usepackage{enumerate}

%% HYPERREF %%
\usepackage{hyperref}
	\hypersetup{
		colorlinks=true,
		linkcolor=blue,
		filecolor=magenta,
		urlcolor=cyan,
	}

%% for code listings:
%\lstset{
%	language=C,
%	basicstyle={\small\ttfamily}
%}

\title{TeamSolver Documentation}
\author{Erik Vesterlund}

\begin{document}
\maketitle

\newpage
\begin{abstract}
This document describes the TeamSolver application. %(not the code)
\end{abstract}

\newpage
\tableofcontents
\newpage

\section{Introduction}

This document outlines the features of the TeamSolver application; code documentation can be read elsewhere.

\section{Command Line Interface}

The command line interface offers no "proper" mode of exit, instead the user is informed of the option to type Ctrl+C (\textasciicircum C) to exit. The two available commands are \texttt{new} and \texttt{load}. If neither is typed, the terminal will output an error message and ask again for a command.

Currently only one command can be used per session; if the user first wants to load a shift and do work on and then create a new shift to do other work on, he will have to restart the program.

\subsection{\texttt{new}}

The command \texttt{new} is used to create a new list of employees (a shift) for possible further manipulation. The user is first asked to input a number, representing the number of employees which will be entered, and then to input the information for that many employees, which consist of a name and a list of competencies, separated by a comma.

If the input passes the format standards described below, the string of competencies if formatted such that the result contains only uppercase alphabetic letters. An instance of the \texttt{Employee} class is then formed from the name- and competencies strings and added to the list of employees: employees who can drive are added to the beginning of the list, employees who cannot are added to the end of the list.

\subsubsection{Overall Format}

The input shall consist of exactly two fields separated by a single comma; more than one comma will be interpreted as there being more than two fields and the input will be rejected.

\subsubsection{Name Format}

Any printable characters may be used to form a name, but at least two of the characters in the name must be alphabetic letters.

\subsubsection{Competence Format}

The competencies string must consist solely of spaces and alphabetic letters; there must at least one alphabetic letter in this string.

\subsection{\texttt{load}}

\section{Graphical Interface}

\section{Solver}

\end{document}




































